\chapter{APOIO AO DISCENTE}

Preocupada com a qualidade social da formação, a UFRPE promove ações e programas de apoio estudantil buscando garantir a igualdade de oportunidades, a melhoria do desempenho acadêmico e, por conseguinte, combater as situações de retenção e evasão. Neste sentido, a Política de Assistência Estudantil desta Instituição tem como propósitos basilares:

\begin{enumerate}
    \item Democratizar as condições de permanência dos jovens na educação superior pública federal;
    \item Minimizar os efeitos das desigualdades sociais e regionais na permanência e conclusão da Educação Superior;
    \item Reduzir as taxas de retenção e evasão;
    \item Contribuir para a promoção da inclusão social por meio da educação.
	Diante do exposto, é exibido no no Quadro 18 alguns programas institucionais de apoio ao estudante da UFRPE.
\end{enumerate}

\begin{center}
  
  \begin{scriptsize}
    \begin{longtable}{p{3.2cm}p{3cm}p{8cm}}
      \caption{\label{quadro:programas-apoio-estudantil-ufrpe}Programas de Apoio Estudantil da UFRPE.} \\
      \toprule
      \textbf{Programa} & \textbf{Resolução} & \textbf{Descrição}\\ \midrule
      Apoio ao ingressante & Res. CEPE/UFRPE nº 023/2017 & Voltado aos alunos ingressantes nos cursos de graduação presencial, regularmente matriculados, e em situação de vulnerabilidade socioeconômica.\\ \midrule
      Apoio ao Discente & Res. CEPE/UFRPE nº 021/2017 & Voltado aos alunos de primeira graduação, regularmente matriculados em cursos de graduação presenciais, e estarem em situação e vulnerabilidade socioeconômica. As bolsas contemplam: \newline
      1. Apoio Acadêmico; \newline
      2. Auxílio Transporte; \newline
      3. Auxílio Alimentação. \\ \midrule
      Apoio à Gestante & Res. CEPE/UFRPE nº 112/2014 & Para as discentes que tenham um filho no período da graduação. Duração máxima: 3 anos e 11 meses. \\ \midrule
      Auxílio Moradia & Res. CEPE/UFRPE nº 062/2012 & Para os estudantes de graduação, de cursos presenciais, regularmente matriculados, residentes fora do município de oferta do curso, reconhecidamente em situação de vulnerabilidade socioeconômica durante a realização da graduação. \\ \midrule
      Auxílio \newline Recepção/Hospedagem & Res. CEPE/UFRPE nº 081/2013 & Para discentes provenientes dos programas de Cooperação Internacional \\ \midrule
      Ajuda de Custo & Res. CEPE/UFRPE nº188/2012 & Destinado a cobrir parte das despesas do aluno com inscrição em eventos científicos, aquisição de passagens, hospedagem e alimentação. \\ \midrule
      Auxílio Manutenção & Res. CEPE/UFRPE nº 027/2017 & Objetiva promover a permanência de alunos residentes, em situação de vulnerabilidade socioeconômica, durante a realização do curso de graduação. \\ \midrule
      Ajuda de Custo para Jogos Estudantis & Res. CEPE/UFRPE nº 184/2007 & Destinado a cobrir despesas com aquisição de passagens e, excepcionalmente, aluguel de transporte coletivo, hospedagem e alimentação para a participação em jogos estudantis estaduais, regionais e nacionais.\\ \midrule
      Promoção ao Esporte & Res. CEPE/UFRPE nº109/2016 & Para estudantes de primeira graduação presencial, regularmente matriculados no curso e na Associação Atlética Acadêmica e que apresentem situação de vulnerabilidade econômica.\\
    \bottomrule
\end{longtable}
\end{scriptsize}
\end{center}

Além da relação constante no Quadro supracitado, são disponibilizados, através da PREG, os seguintes Programas: Atividade de Vivência Interdisciplinar – PAVI, Monitoria Acadêmica, PET e Incentivo Acadêmico – BIA. No que diz respeito à oferta de bolsas de iniciação científica e de extensão, estas são, respectivamente, viabilizadas pela Pró-Reitoria de Pesquisa e Pós-Graduação – PRPPG e a Pró-Reitoria de Extensão – PRAE, ambas vinculadas a projetos de pesquisa e extensão da UFRPE.

Destaca-se, ainda, que a Pró-Reitoria de Gestão Estudantil e Inclusão – PROGESTI dispõe de plantão psicológico para atendimento aos discentes da Instituição, além de acompanhamento pedagógico com o objetivo de auxiliar o estudante em seu processo educacional através de um planejamento individualizado de ações específicas de aprendizagem.

Já a Assessoria de Cooperação Internacional – ACEI, estabelecida em 2007, tem a finalidade de ampliar e consolidar a internacionalização e os laços de cooperação interinstitucional da Universidade, proporcionando à comunidade acadêmica oportunidades de usufruir da mobilidade como forma de fortalecer o desempenho acadêmico e fomentar experiências culturais.

O curso de Bacharelado em Ciência da Computação possuirá uma Comissão de Orientação e Acompanhamento Acadêmico – COAA com o objetivo de acompanhar e orientar os estudantes em situação de insuficiência de rendimento, conforme a Resolução CEPE/UFRPE nº 154/2001. A COAA é composta pelo Coordenador do Curso, 3 (três) professores e 1 (um) estudante, indicados pela Coordenação e homologada pelo CCD. 
