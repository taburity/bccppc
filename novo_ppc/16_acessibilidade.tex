\chapter{ACESSIBILIDADE}

A Resolução N° 090/2013 é o documento inicial dentro da UFRPE que institui o NACES (Núcleo de Acessibilidade), na Sede de nossa Universidade, que é composto pela Coordenação do Núcleo, pelos Setores de Acessibilidade espalhados pelas Unidades Acadêmicas, e Comissão de Acessibilidade – grupo de representantes da comunidade acadêmica – que atua de forma propositiva.

As ações, tanto do NACES bem como dos Setores de Acessibilidade, foram delineadas com a Resolução Nº 172/2013 de 03 de setembro, caracterizando assim sua natureza e finalidade:

\begin{citacao}
    Art. 1º – O Núcleo de Acessibilidade (NACES) da Universidade Federal Rural de Pernambuco é um órgão executivo da Administração Superior, diretamente subordinado à Reitoria e tem por finalidade atender aos discentes, docentes, técnico-administrativos e terceirizados com deficiência ou com mobilidade reduzida, quanto ao seu acesso e permanência na Universidade Federal Rural de Pernambuco (UFRPE), promovendo e desenvolvendo ações que visem eliminar ou minimizar barreiras físicas, atitudinais, pedagógicas e na comunicação e informação que restringem a participação, a autonomia pessoal e do desenvolvimento, social e profissional.
\end{citacao}

Ainda segundo a Resolução Nº 172/2013, os setores de acessibilidade da UFRPE deverão ser compostos por uma equipe multidisciplinar, formada por: Psicólogo, Brailista, Pedagogo e Tradutor/Intérprete de LIBRAS – Língua Brasileira de Sinais. Atualmente, \textbf{o Setor de Acessibilidade da UAG é formado por uma Pedagoga e dois Tradutores/Intérpretes de LIBRAS}.

O Setor de Acessibilidade, \textbf{atua com o objetivo de mapear/localizar e intervir nas demandas existentes}, quando as demandas extrapolam as atribuições dos servidores lotados, as mesmas serão encaminhadas à Coordenação do NACES, para que sejam dados os devidos encaminhamentos. Utiliza-se como ferramentas de trabalho para a identificação de demandas {\bfseries \textit{Formulários de Acessibilidade}} (para docentes, discentes, técnicos e terceirizados), no caso dos discentes, no ato da matrícula é disponibilizado o formulário para fins de atualização dos dados do setor e posterior intervenção.

As demandas do setor emergem tanto da comunidade da UAG (direções, demais setores, docentes, discentes, técnicos e terceirizados) como da própria sede através do NACES.

Finalmente, ressalta-se que o Setor de Acessibilidade da UAG tem como atribuição dirimir possíveis dúvidas e disponibilizar informações acerca da temática “acessibilidade”, bem como temas que perpassam por ela, a fim de quebrar tabus e mitos.

\section{Acessibilidade Pedagógica}

A Resolução 172/2013, especificamente em seu Art. 4º, que dispõe sobre a organização e composição dos Setores de Acessibilidade das Unidades, prevê a figura do Pedagogo, para compor a equipe de profissionais responsáveis pelas ações ligadas ao processo de inclusão e acessibilidade na Unidade Acadêmica de Garanhuns - UAG.

O planejamento das atividades específicas, devido ao caráter desafiador do Perfil da Acessibilidade Pedagógica definido na própria Resolução, tem demandado tempo para sua estruturação e, vem ocorrendo mediante estudo e pesquisa em literatura específica, legislação, reunião e discussão com os profissionais que compõem o NACES, sem perder de vista o Regimento interno do Núcleo de Acessibilidade da UFRPE, e por fim o teor da descrição sumária dos cargos técnico-administrativos em educação, disponibilizada pela Gestão de Pessoas dessa Instituição, especificamente o de Pedagogo.

Assim sendo, o Serviço de Acessibilidade Pedagógica, através do desenvolvimento das ações abaixo relacionadas, (que poderão ser redefinidas de acordo com as demandas), integrado com outros serviços do Setor, participará na efetivação da política de acessibilidade da UAG/UFRPE.

\begin{center}
  
  \begin{scriptsize}
    \begin{longtable}{p{5cm}p{9cm}}
      \caption{\label{quadro:acoes-pedagogicas}Ações pedagógicas.}\\
      \toprule
      \textbf{Responsável} & \textbf{Ações específicas}\\ 
        \midrule
        Maria Gorete Rodrigues de Siqueira \newline Pedagoga \newline \href{mailto:acessibilidadepedagogica@ufrpe.br}{acessibilidadepedagogica@ufrpe.br} 
        & - Mapeamento dos discentes com demanda de acessibilidade pedagógico;\\
        & - Entrevistas e/ou aplicação de questionários aos discentes com demandas de acessibilidade pedagógica para levantamento das necessidades;\\
        \addlinespace
        & - Acompanhamento pedagógico do desempenho acadêmico dos estudantes com deficiência e/ou mobilidade reduzida cadastrados no Setor através de acesso ao Sig@;\\
        \addlinespace
        & - Realização de reuniões semestrais com discentes cadastrados no Setor, além de atendimentos individualizados;\\
        \addlinespace
        & - Encaminhamentos das demandas a outros profissionais e/ou serviço;\\
        \addlinespace
        & - Reunião com coordenadores e professores no sentido de orientar sobre as necessidades didático-pedagógicas do discente;\\
        \addlinespace
        & - Organização de acervo bibliográfico local (Setor de Acessibilidade) com a temática específica;\\
        \addlinespace
        & - Participação em projetos de pesquisa e extensão sobre a temática de educação inclusiva;	\\
        \addlinespace
        & - Participação em Seminários e Cursos relacionados a temática de educação inclusiva;		\\
        \addlinespace
        & - Trabalho colaborativo com os outros profissionais.\\
        
      \bottomrule
\end{longtable}
\end{scriptsize}      
\end{center}

\section{Acessibilidade Comunicacional em LIBRAS}

Texto extraído da Dissertação “\textbf{A IDENTIDADE DO PROFISSIONAL TRADUTOR E INTÉRPRETE DE LÍNGUA BRASILEIRA DE SINAIS – LIBRAS}: das suas concepções às suas práticas” de Carvalho, 2015, no que toca a página 25, do Capítulo II).

\begin{citacao}
    O tradutor/intérprete atua na fronteira entre os sentidos da língua de origem e da língua alvo, com os processos de interpretação relacionando-se com o contexto no qual o signo é formado. [\ldots]. A interpretação é um processo ativo, que procede de sentidos que se encontram, existindo, apenas, na relação entre eles, como um elo nessa cadeia de sentidos (LACERDA, 2009, p. 8).
\end{citacao}

Tratando da relevância e complexidade do trabalho do intérprete, Lima (2006) aponta que os intérpretes de língua de sinais, são profissionais que não trabalham apenas com a língua utilizada pela comunidade surda, vão mais além, e interpretam também o que ocorre no âmbito da expressão desta língua: a cultura, a história, os movimentos sociais e políticos.

A tradução de um texto de português para LIBRAS apresenta variáveis distintas das realizadas entre as línguas orais, por exemplo, as modalidades das línguas envolvidas. O português, língua de modalidade oral - auditiva, é expressa através dos sons e decodificada pela audição, enquanto que a LIBRAS, língua de modalidade gestual - visual, é produzida por movimentos, principalmente das mãos e decodificada pelos olhos. Sobre este assunto Segala conclui:

\begin{citacao}
    (\ldots) para traduzir os textos como língua-fonte, Português brasileiro, para Língua Brasileira de Sinais – Libras, o tradutor deve ter domínio em Língua Portuguesa e Libras; suas variações linguísticas, sociais e culturais (bilíngues- bimodais), e também ter conhecimento do tema, ou seja, da área e suas normas linguístico- culturais. A Língua de chegada (Libras) deve ser clara e moderna, e utilizar os sinais mais comuns aos surdos, os usuários de Libras, não seguindo a estrutura da Língua Portuguesa, nunca traduzindo literalmente palavra por sinais, obedecendo a ordem dos parágrafos sem a necessidade de se preocupar com virgulação, e sendo fiel ao sentido dos textos para Libras, principalmente para que os usuários de Libras entendam e possam interpretar os textos em Libras (SEGALA, 2010, p. 57).
\end{citacao}


\begin{center}
  
  \begin{scriptsize}
    \begin{longtable}{p{5cm}p{9cm}}
      \caption{\label{quadro:acoes-promocao-acessibilidade}Ações de promoção à acessibilidade.}\\
      \toprule
      \textbf{Responsável} & \textbf{Ações específicas}\\ 
        \midrule
        Geyson Lima de Carvalho \newline Núbia Poliane C. T. Pires de Lima \newline acessibilidade.uag@ufrpe.br
        & - Promover acessibilidade comunicacional no que tange ao par linguístico Língua Portuguesa e LIBRAS em eventos, aulas e atividades promovidos pela UFRPE em que haja a demanda (usuários que necessitem do serviço de tradução/interpretação);\\
        \addlinespace
        & - Tornar acessível em LIBRAS documentos, editais e outros materiais de interesse da UFRPE;\\
        
      \bottomrule
\end{longtable}
\end{scriptsize}      
\end{center}

