\chapter{POLÍTICAS INSTITUCIONAIS NO ÂMBITO DO CURSO}

Entre os diversos espaços de construção do conhecimento, a Universidade é um lugar privilegiado de desenvolvimento humano, científico-tecnológico e social. Contudo, a qualidade da educação e o sucesso dos profissionais formados pelas universidades dependem, em grande medida, do nível de interação e articulação entre os três pilares balizadores da formação universitária: o ensino, a pesquisa e extensão.

Partindo do entendimento de que estas atividades precisam atuar de forma complementar e interdependente, este PPC está em sintonia com o PPI da UFRPE. O PPI integra o PDI UFRPE 2013-2020, atualizado pela comunidade acadêmica entre 2016 e 2017. A estrutura e as diretrizes para a elaboração do PDI passaram a ser definidas pelo Decreto nº 9.235/2017 (BRASIL, 2017). Neste contexto, as diretrizes das políticas institucionais no âmbito do ensino, pesquisa e extensão, preconizadas no PPI e com as quais o curso dialoga de forma mais estreita, são as seguintes:

\begin{itemize}
	\item Interação e organicidade entre as modalidades de ensino presencial e à distância;
	\item Implantação de metodologia de ensino híbrido;
	\item Apoio e incentivo à elaboração de material didático adequado para a EAD.
\end{itemize}
 	
As modalidades de ensino presencial e a distância não são concebidas de forma dicotômica, mas complementares em um mesmo planejamento didático. Tal aspecto se traduz tanto pela concepção híbrida do processo de ensino e aprendizagem presente na metodologia e avaliação (seção 14), quanto pelo suporte promovido por equipe multiprofissional ao desenvolvimento e acompanhamento das atividades semipresenciais e a distância (seção 9.1).

\begin{itemize}
	\item Políticas de permanência nos cursos de graduação;	
	\item Elevação da taxa de sucesso, com ações de combate à evasão e ao abandono;
	\item Política de acompanhamento 	do estudante egresso.
\end{itemize}
 
Uma formação de qualidade não está dissociada da existência de determinadas condições sociais, econômicas e pedagógicas necessárias ao desenvolvimento do estudante durante o curso. Em nível institucional, os programas da UFRPE descritos na seção 15, oferecem suporte ao estudante no que tange aos mais variados aspectos, desde alimentação até bolsas de manutenção acadêmica e iniciação à pesquisa, além do estímulo a atividades de extensão. O acompanhamento sistemático do desempenho acadêmico do aluno também será objeto de atenção, de modo a identificar, prematuramente, demandas por um apoio pedagógico e/ou psicológico mais próximo. Tal acompanhamento ocorrerá por meio da COAA, bem como por meio de autoavaliações periódicas no âmbito do curso (seção 18). No caso do estudante egresso, o curso estabelecerá articulações com a Coordenação de Acompanhamento e Monitoramento de Egressos - CAME, de modo a fomentar formações, encontros e seminários sobre o universo profissional do bacharel em ciência da computação. A partir da primeira turma formada, o curso utilizará os relatórios da CAME em seu processo de autoavaliação e planejamento.

\begin{itemize}
	\item Promoção de estratégias que levem ao avanço nos indicadores de qualidade dos cursos de graduação;
	\item Formação continuada dos docentes a partir das necessidades de suas áreas específicas de formação e didático-pedagógicas;
	\item Oferta de formação continuada a técnico-administrativos, tutores e coordenadores de curso.
\end{itemize}

Considerando que na definição da qualidade do curso concorrem diversos fatores, o planejamento e a autoavaliação sistemáticos proporcionarão a elaboração de planos de ações que apontarão aspectos a serem corrigidos e aprimorados, conforme exposto na seção 16. No caso da formação docente, observa-se que esta já é uma prática estabelecida pela UFRPE, através dos cursos de atualização didático-pedagógica (Resolução CEPE/UFRPE nº 211/2009). No âmbito do curso será proposta, em parceria com a PREG, uma formação específica para os professores de Ciência da Computação, considerando o trabalho com o Ensino Híbrido e a PBL. Também serão promovidas formações para os tutores, o coordenador do curso e membros da equipe multidisciplinar.	 
	 	
\begin{itemize}
	\item Estímulo à produção científica e tecnológica; 
	Fomento à construção e à socialização de tecnologias, incluindo as sociais, a fim de promover a sustentabilidade de comunidades localizadas na zona rural do estado de Pernambuco; 	
	Promoção da extensão enquanto processo educativo, cultural e científico que articulem ensino e pesquisa, integrando as várias áreas do conhecimento e aproximando diferentes sujeitos sociais com vistas à construção de uma sociedade igualitária e justa.
\end{itemize}

O envolvimento com a pesquisa, em nível de graduação, constitui elemento importante na formação do bacharel em ciência da computação, quando consideramos o seu perfil profissional (Seção 6). A inserção na prática da pesquisa ocorrerá tanto em nível de programas de iniciação científica, como o PIBIC, quanto por meio do desenvolvimento de projetos interdisciplinares. A pesquisa também se apresenta como um aspecto do processo de ensino e aprendizagem (seção 14). A extensão, no momento em que dialoga com as demandas sociais, econômicas e culturais da região, propicia aos estudantes o envolvimento com realidades diversas e o desenvolvimento de soluções para os problemas demandados pelos diversos atores sociais. O desenvolvimento de práticas articuladas de ensino, pesquisa e extensão também encontram no PET um espaço profícuo para a sua realização, contribuindo, assim, para uma formação mais orgânica do futuro profissional.	 	 	

\begin{itemize}
	\item Promoção de eventos acadêmicos;
	\item Intensificação do envolvimento da instituição na participação e organização de eventos científicos, educativos, artísticos e culturais locais, regionais, nacionais e internacionais;
\end{itemize}

O curso estimulará a realização de eventos acadêmicos, bem como a participação dos estudantes em seminários, encontros e congressos. Observe-se que a UFRPE dispõe de um evento anual onde os alunos poderão apresentar os resultados de suas pesquisas e atividades; trata-se da Jornada de Ensino, Pesquisa e Extensão – JEPEX. Considera-se que a participação em tais eventos integra a formação dos bacharéis. No âmbito do curso, tais eventos poderão integrar o planejamento anual das atividades.	 	 	

\begin{itemize}
	\item Estímulo à cultura do empreendedorismo econômico e social na instituição através do fortalecimento das ações das incubadoras existentes (INCUBACOOP e INCUBATEC), da ampliação dos editais e da promoção de novas incubadoras; 	
	\item Ampliação do diálogo da Universidade com setores da iniciativa pública e privada em geral, a fim de intensificar ações de extensão em regime colaborativo;
\end{itemize}

Considerando os objetivos do curso expressos na Seção 4, o empreendedorismo integra a formação do Bacharel em Ciência da Computação. Neste sentido, o curso, em articulação com o Núcleo de Relações Institucionais e Convênios – NURIC, deverá buscar parcerias com instituições públicas, privadas e sem fins lucrativos. Do mesmo modo, serão estimulados a realização de projetos e eventos que aproximem os estudantes da cultura do empreendedorismo econômico e social, com especial atenção, ao contexto regional da UAG.	 	 	

\begin{itemize}
	\item Reforço das ações de promoção dos valores democráticos, da justiça social e da liberdade, de garantia de direitos sociais e individuais e do combate a toda forma de discriminação – étnica, de gênero, geracional, social, sexual, religiosa, entre outras;
	\item Compromisso com a educação de qualidade, inclusiva e acessível a todos.
\end{itemize}

Considerando os objetivos do curso expressos na Seção 4, a formação do Bacharel em Ciência da Computação não prescindirá de uma discussão acerca da promoção dos valores democráticos, justiça social, direitos humanos e luta contra a discriminação. Esse debate ocorrerá de forma transversal no currículo, além de estar presente em eventos, ações e projetos. A preocupação com o combate a toda forma de discriminação contemplará ações de inclusão, garantindo a valorização das diferenças e o atendimento às pessoas com necessidades educacionais especiais (seções 14 e 16). Neste âmbito, uma das estratégias adotadas pelo curso, em parceria com o NACES e o Setor de Acessibilidade da UAG, será o de fomentar o desenvolvimento de projetos de ensino, pesquisa e extensão voltados à promoção da Acessibilidade.
