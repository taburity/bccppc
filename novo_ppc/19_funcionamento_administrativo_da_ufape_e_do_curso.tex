\chapter{FUNCIONAMENTO ADMINISTRATIVO DA UAG E DO CURSO}
	
A UAG, como Unidade Universitária da UFRPE, desenvolve atividades administrativas, didático-científicas e extensionistas que congrega servidores, estudantes e membros da comunidade local. O \textit{Estatuto e Regimento Geral das Unidades Acadêmicas Fora de Sede}, com alteração promovida pela Resolução CONSU/UFRPE nº 003/2017, estabelece as bases para a organização administrativa da UAG.

De acordo com o \textit{Estatuto}, a mesma é exercida pela Diretoria Geral e Acadêmica, Diretoria Administrativa, pelo Conselho Técnico-Administrativo e Secretaria (Art. 3º). A Diretoria Geral e Acadêmica, que é exercida pelo Diretor Geral e Acadêmico e seu substituto eventual, coordenando as atividades acadêmicas e fiscalizando as atividades administrativas da Unidade (Art. 4º). Como prevê o \textit{Regimento Geral} (Art. 2º), o Diretor Geral e Acadêmico tem as seguintes atribuições:

\begin{enumerate}
    \item Participar, como membro nato, do Conselho Técnico-Administrativo da Unidade Acadêmica;
    \item Administrar e representar a Unidade Acadêmica;
    \item Convocar e presidir as reuniões do Conselho Técnico Administrativo da Unidade Acadêmica;
    \item Cumprir e fazer cumprir as deliberações do Conselho Técnico Administrativo da Unidade Acadêmica e dos órgãos deliberativos da Administração Superior, 	bem como as instruções dos demais órgãos executivos da esfera administrativa;
    \item Cumprir e fazer cumprir as 	disposições do Estatuto da UFRPE, do Regimento Geral da UFRPE e do 	Regimento da Unidade Acadêmica;
    \item Submeter, na época devida, 	conforme instruções dos órgãos superiores, à consideração do Conselho Técnico-Administrativo da Unidade, o plano de atividades de cada período letivo, inclusive a oferta de disciplinas;
    \item Planejar e submeter à aprovação do Conselho Técnico-Administrativo da Unidade a distribuição dos encargos de ensino, pesquisa e extensão, respeitando, dentro do possível, as preocupações científico-culturais dos docentes;
    \item Fiscalizar a assiduidade dos docentes e dos técnico-administrativos da Unidade Acadêmica; 	
    \item Fiscalizar a observância do 	regime acadêmico, o cumprimento dos programas de ensino e a execução dos demais planos de trabalho;
    \item Apresentar, no fim de cada período letivo, após aprovação do Conselho Técnico-Administrativo da Unidade, o relatório de atividades da Unidade, sugerindo as providências cabíveis para a maior 	eficiência do ensino, da pesquisa e da extensão;
    \item Adotar, em casos de urgência, medidas que se imponham em matéria de competência do Conselho da 	Unidade, ad referendum, submetendo-as à ratificação deste, no prazo de cinco (5) dias;
    \item Integrar o Conselho Universitário;
    \item Submeter os casos omissos no Regimento da Unidade Acadêmica ao Conselho Técnico-Administrativo da Unidade.
\end{enumerate}

A Diretoria Administrativa, segundo o \textit{Estatuto}, supervisiona e coordena os serviços administrativos da Unidade Acadêmica, executados pelos Setores de Pessoal, de Contabilidade e Finanças, de Informática, de Patrimônio, de Comunicação, de Material e de Serviços Gerais, e de Assistência Social à Comunidade Universitária (Art. 7º). Nas faltas e impedimentos do Diretor Administrativo, a Diretoria será exercida pelo seu respectivo substituto eventual (Art. 8, Parágrafo Único).

O Diretor Administrativo de Unidade Acadêmica, em conformidade com o \textit{Regimento Geral} (Art. 3º), tem as atribuições definidas abaixo:

\begin{enumerate}
    \item Participar, como membro nato, do Conselho Técnico-Administrativo da Unidade Acadêmica;
    \item Observar o cumprimento da 	Política definida pelo Conselho Universitário referente a área administrativa;
    \item Contribuir para integração dos diversos setores da Unidade Acadêmica na área administrativa;
    \item Realizar estudos sobre a estrutura e procedimentos, propondo medidas que visem maior eficiência das atividades administrativas da Unidade Acadêmica;
    \item Desempenhar outras atribuições de caráter permanente, temporário ou ocasional, delegadas pelo Diretor Geral e Acadêmico;
    \item Emitir parecer sobre assuntos relacionados com sua área de competência;
    \item Coordenar a gestão do pessoal técnico-administrativo lotado na Unidade Acadêmica e solicitar as substituições que se fizerem necessárias;
    \item Enviar ao Reitor, em tempo 	hábil, a discriminação da receita e da despesa da Unidade, como subsídio à elaboração da proposta orçamentária; 
    \item Pronunciar-se sobre a escala de férias dos técnico-administrativos, resguardando o andamento 	normal das atividades de ensino, pesquisa e extensão da Unidade  Acadêmica, encaminhando as proposições ao Diretor Geral e Acadêmico.
\end{enumerate}

O Conselho Técnico Administrativo da Unidade Acadêmica, como previsto no Art. 10º do Estatuto, é constituído pelos seguintes membros titulares e respectivos suplentes:

\begin{enumerate}
    \item Diretor Geral e Acadêmico, como presidente;
    \item Diretor Administrativo; 	
    \item Coordenador Geral dos Cursos de Graduação; 	
    \item Um representante dos professores titulares;
    \item Dois representantes dos professores associados;	
    \item Três representantes dos professores adjuntos; 	
    \item Três representantes dos professores assistentes; 	
    \item Um representante dos professores auxiliares; 	
    \item Dois representantes dos técnico-administrativos; 	
    \item Dois representantes dos discentes.
\end{enumerate}

Os representantes referidos nas alíneas d, e, f, g, h, i e j são escolhidos dentre os seus pares, conforme normas elaboradas pelo Conselho Técnico Administrativo da Unidade Acadêmica, com mandato de dois anos, conforme legislação vigente (Art. 10 § 1º).

Conforme consta no \textit{Regimento Geral} (Art. 1º), o Conselho Técnico Administrativo da Unidade Acadêmica, como órgão consultivo, normativo e deliberativo de cada Unidade Acadêmica, terá como atribuições:

\begin{enumerate}
    \item Aprovar a distribuição das tarefas de ensino, pesquisa, extensão e outros, entre os docentes que integram a Unidade Acadêmica, conciliando, dentro do possível, os interesses da Unidade com as preocupações científico-culturais dominantes dos referidos docentes;
    \item Aprovar, nos limites de sua competência, os projetos de pesquisa ou planos de cursos de especialização, aperfeiçoamento e extensão da Unidade elaborados por especialistas da Unidade Acadêmica;
    \item Elaborar e fiscalizar o plano de aplicação de recursos da Unidade;
    \item Apreciar e julgar as propostas de alterações do quadro de pessoal docente ou administrativo da Unidade, para encaminhamento às autoridades superiores;
    \item Elaborar e aprovar, para o devido e tempestivo encaminhamento ao Pró-Reitor de Ensino de Graduação, a lista de disciplinas da Unidade Acadêmica que podem ser ofe­rtadas e ministradas em cada semestre, com o respectivo número de turmas e de vagas;
    \item Adotar providências para o 	constante aperfeiçoamento do pessoal docente e 	técnico-administrativo da Unidade;
    \item Emitir pareceres em assuntos de sua competência;
    \item Promover e exercer as atribuições de sua competência, nos concursos para docentes;
    \item Assessorar o Diretor Geral e 	Acadêmico; 
    \item Organizar, em reunião específica, a lista tríplice para escolha da nomeação do Diretor Geral e Acadêmico;
    \item Exercer as demais atribuições que se incluam, de maneira expressa ou implícita, no âmbito de sua 	competência.
\end{enumerate}

No que se refere à Secretaria, esta dá suporte administrativo à Unidade Acadêmica. A chefia da Secretaria é exercida por técnico-administrativo indicado pelo Diretor Geral e Acadêmico e designado pelo Reitor. (\textit{Estatuto}, Art. 13º). A coordenação didática dos cursos da UAG é exercida por um Colegiado Geral de Coordenação Didática - CGCD, constituído pelo Coordenador Geral dos Cursos, pelos Coordenadores de Curso, por um docente de cada curso, e por dois (2) representantes do corpo discente. O CGCD é presidido pelo Coordenador Geral dos Cursos de Graduação ou seu substituto eventual. (\textit{Estatuto}, Art. 14). São atribuições do CGCD, conforme o \textit{Regimento Geral}, Art. 4º:

\begin{enumerate}
    \item Avaliar modificações na matriz curricular elaboradas pelo Colegiado de Curso e propô-las ao Conselho de Ensino, Pesquisa e Extensão;
    \item Avaliar elenco de disciplinas optativas elaborada pelo Colegiado do Curso e propô-las ao Conselho de Ensino, Pesquisa e Extensão;
    \item Promover através de propostas elaboradas em conjunto com a(s) 	Coordenação(ões) de Curso(s) e devidamente justificadas, ao Conselho de Ensino, Pesquisa e Extensão, a melhoria contínua do(s) curso(s).
    \item Propor 	à Câmara competente do Conselho de Ensino, Pesquisa e Extensão,  modificações nos planos dos respectivos cursos;
    \item Estudar e analisar, em cada período letivo, os planos de ensino das 	disciplinas da(s), da (as) matriz (es) curricular(es) do(s) curso(s), proposta(s) pelas respectivas Coordenações de Curso, sugerindo a estas as modificações julgadas necessárias;
    \item Coordenar o processo eletivo para composição da lista tríplice para Coordenador de Curso de Graduação e seu substituto eventual;
    \item Estabelecer a distribuição das tarefas de ensino (aulas, atendimento aos alunos e preparação de aulas), pesquisa, extensão e outros, entre os docentes que integram a Unidade Acadêmica, conciliando, dentro  do possível, os interesses da Unidade com as preocupações 	científico/culturais dominantes dos referidos docentes;
    \item Exercer as demais funções que lhe são deferidas em lei, no Estatuto e neste Regimento;
    \item Deliberar sobre os casos omissos na esfera de sua competência.
\end{enumerate}

O Coordenador Geral dos Cursos de Graduação da UAG tem as seguintes atribuições (Regimento Geral, Art. 5º):

\begin{enumerate}
    \item Participar como membro nato, do Conselho de Ensino, Pesquisa e Extensão da UFRPE;
    \item Participar como membro nato, do Conselho Técnico-Administrativo da Unidade 	Acadêmica;
    \item Convocar e presidir as reuniões do Colegiado Geral de Coordenação Didática;
    \item Representar o Colegiado Geral de Coordenação Didática junto aos órgãos 	deliberativos da Universidade, na forma do Estatuto e deste Regimento;
    \item Encaminhar expediente e processos aprovados no Colegiado Geral de Coordenação Didática;
    \item Adotar, em caso de urgência, providências da competência do Colegiado Geral de Coordenação Didática, ad referendum deste, ao qual as submeterá no prazo de cinco dias;
    \item Coordenar e fiscalizar as atividades dos docentes que integram a Unidade Acadêmica, distribuindo com eles as tarefas didáticas, relativas 	às turmas de alunos matriculados nas disciplinas;
    \item Pronunciar-se 	sobre a escala de férias dos docentes, resguardando o andamento normal das atividades de ensino, pesquisa e extensão da Unidade Acadêmica;
    \item Representar, no Conselho Técnico-Administrativo da Unidade Acadêmica, os  interesses das Coordenações de Curso de Graduação;
    \item Tomar quaisquer outras iniciativas de interesse das Coordenações de Curso de Graduação;
    \item Responder pelas ações de assistência estudantil junto à PROGESTI;
    \item Cumprir e/ou fazer cumprir as determinações do Colegiado Geral de Coordenação Didática e planos dos cursos, da Administração Superior e de seus Conselhos, bem como zelar pelo cumprimento das disposições pertinentes no Estatuto e neste Regimento.
\end{enumerate}

A coordenação didática do curso de graduação é exercida por um Colegiado de Coordenação Didática - CCD, constituído pelo coordenador do curso, como presidente, pelo seu substituto eventual, como vice-presidente, por docentes dos primeiros quatro períodos do curso (quatro representantes) e do quinto ao último período do curso (cinco representantes), que ministram disciplinas no curso, e por representantes do corpo discente de graduação (Estatuto, Arts. 15 e 16). O CCD terá as seguintes atribuições definidas pelo Regimento Geral em seu Art. 6º:

\begin{enumerate}
    \item Elaborar modificações no currículo do curso, propondo-as ao CGCD;
    \item Propor ao CGCD o elenco de 	disciplinas optativas do curso;
    \item Promover, através de propostas devidamente justificadas ao CGCD, a melhoria contínua do curso;
    \item Propor ao CGCD modificações nos planos dos respectivos cursos;
    \item Propor, em cada período letivo, os planos de ensino das disciplinas do currículo do curso;
    \item Apreciar e deliberar sobre as 	solicitações acerca do aproveitamento de estudos e adaptações, ouvidos os docentes da Unidade com competência para julgar e emitir pareceres sobre o conteúdo de tais solicitações;
    \item Exercer as demais funções que lhe são, explícita ou implicitamente, deferidas em Lei, no Estatuto e neste Regimento Geral;
    \item Deliberar sobre os casos omissos na esfera de sua competência.
\end{enumerate}

O coordenador de curso de graduação tem as seguintes atribuições, consoante o Regimento Geral, Art. 7º:

\begin{enumerate}
    \item Convocar e presidir as reuniões do respectivo Colegiado; 	
    \item Representar o Colegiado junto ao CGCD da Unidade, na forma do Estatuto e deste Regimento;
    \item Submeter ao CCD as modificações propostas para o plano ou currículo do 	curso;
    \item Encaminhar expediente e processos aprovados no CCD;
    \item Coordenar e fiscalizar a execução dos planos e a programação do respectivo curso, tomando as medidas adequadas ou propondo-as aos órgãos competentes; 
    \item Adotar em caso de urgência, 	providências da competência do Colegiado, ad 	referendum deste, ao 	qual as submeterá no prazo de cinco dias;
    \item Atuar junto ao CGCD e Diretoria Geral e Acadêmica, traçando as normas que conduzem à gestão racional e objetiva do curso do qual está representando;
    \item Cumprir e/ou fazer cumprir as determinações do CCD e plano do curso o qual representa, da Administração Superior e de seus Conselhos, do CGCD, bem como zelar pelo cumprimento das disposições pertinentes no Estatuto e neste Regimento.
\end{enumerate}

\section{Atuações do Núcleo Docente Estruturante - NDE}

Regulamentado pela Resolução/UFRPE nº065/2011 e Resolução/CONAES nº 01, de 17 de junho de 2010, o NDE é o órgão consultivo responsável pela concepção, atualização e revitalização do Projeto Pedagógico do Curso. Ele é constituído por, no mínimo, cinco professores pertencentes ao corpo docente do curso, além do Coordenador do Curso que exerce a função de presidente. Dos que compõem o NDE, no mínimo, 25\% devem ter titulação de doutor, e ao menos 20\% devem possuir regime de dedicação exclusiva. São atribuições do NDE, entre outras (Resolução/UFRPE nº065/2011, Art. 3º):

\begin{enumerate}
    \item Estabelecer o perfil profissional do egresso do curso; 	
    \item Atualizar periodicamente o projeto pedagógico do curso; 	
    \item Conduzir os trabalhos de reestruturação curricular, para aprovação no Colegiado de Curso, sempre que necessário;
    \item Supervisionar as formas de avaliação e acompanhamento do curso definidas pelo Colegiado; 	
    \item Analisar e avaliar os Planos de Ensino dos componentes curriculares;
    \item Zelar pela integração curricular interdisciplinar entre as diferentes atividades de ensino constantes no currículo; 	
    \item Indicar formas de incentivo ao desenvolvimento de linhas de pesquisa e extensão, oriundas de necessidades da graduação, de exigências do mercado de trabalho e afinadas com as políticas públicas relativas à área de conhecimento do curso; 	
    \item Zelar pelo cumprimento das Diretrizes Curriculares Nacionais para os Cursos de Graduação.
\end{enumerate}

Ao Presidente do Núcleo compete:

\begin{enumerate}
    \item Convocar e presidir reuniões, com direito a voto, inclusive o de qualidade;
    \item Representar o NDE junto aos 	órgãos da instituição; 	
    \item Encaminhar as deliberações do Núcleo; 	
    \item Designar relator ou comissão 	para estudo de matéria a ser decidida pelo Núcleo e um representante do corpo docente para secretariar e lavrar as atas;
    \item Coordenar a integração com os demais colegiados e setores da Universidade.
\end{enumerate}
