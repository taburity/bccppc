\chapter{ENQUADRAMENTO DO CURSO À LEGISLAÇÃO VI\-GEN\-TE}

Considerando os dispositivos legais que regulamentam o funcionamento do Curso de Bacharelado em Ciência da Computação, este PPC foi construído, coletivamente, sob a égide das leis, decretos, resoluções e pareceres, detalhados a seguir:


\begin{center}
    
    \begin{scriptsize}
        \begin{longtable}{@{}lp{8.7cm}}
            \caption{\label{quadro:base-legal-geral-curso}Base legal geral do Curso de Bacharelado em Ciência da Computação.}\\
    \toprule
    \textbf{Lei, Decreto, Resolução e Parecer} & \textbf{Escopo} \\ 
    \midrule
    Lei nº 9.394/1996 & Estabelecer as diretrizes e bases da educação nacional.\\ \midrule
    Lei nº 13.005/2014	& Aprovar o Plano Nacional de Educação- PNE \\ \midrule
    Lei nº 11.645/2008	& Alterar a Lei 9.394, de 20 de dezembro de 1996, modificada pela Lei nº 10.639, de 9 de janeiro de 2003, que estabelece as diretrizes e bases da educação nacional, para incluir no currículo oficial da rede de ensino a obrigatoriedade da temática ``História e Cultura Afro-Brasileira e Indígena''. \\ \midrule
    Lei nº 12.764/2012 & Instituir a Política Nacional de Proteção dos Direitos da Pessoa com Transtorno do Espectro Autista. \\ \midrule
    Lei nº 13.146/2015 & Instituir a Lei Brasileira de Inclusão da Pessoa com Deficiência (Estatuto da Pessoa com Deficiência). \\ \midrule
    Lei nº 9.795/1999 & Dispor sobre a educação ambiental, institui a Política Nacional de Educação Ambiental e dá outras providências. \\ \midrule
    Decreto nº 5.296/2004 & Estabelecer normas gerais e critérios básicos para a promoção da 	acessibilidade das pessoas portadoras de deficiência ou com mobilidade reduzida. \\ \midrule
    Decreto nº 5.626/2005 & Dispor sobre o Ensino da Língua Brasileira de Sinais - LIBRAS. \\ \midrule
    Resolução CNE/CES nº 2/2007 & Dispor sobre carga horária mínima e procedimentos relativos à integralização e duração dos cursos de graduação, bacharelados, na modalidade presencial. \\ \midrule
    Resolução CONFEA nº 218/73 & Discriminar as atividades das diferentes modalidades profissionais da Engenharia, Arquitetura e Agronomia em nível superior e em nível médio, para fins da fiscalização de seu exercício profissional. \\ \midrule
    Resolução CNE/MEC nº 1/2012 & Estabelecer Diretrizes Nacionais para a Educação em Direitos Humanos.\\ \midrule
    Resolução CNE/MEC nº 2/2012 & Estabelecer as Diretrizes Curriculares Nacionais para a Educação Ambiental. \\ \midrule
    Resolução CNE/CES nº 5/2016	& Instituir as Diretrizes Curriculares Nacionais para os cursos de graduação na área da Computação, abrangendo os cursos de bacharelado em Ciência da Computação, em Sistemas de Informação, em Engenharia de Computação, em Engenharia de Software e de licenciatura em Computação, e dá outras providências.\\ \midrule
    Resolução CNE/MEC nº 1/2004	& Instituir as Diretrizes Curriculares Nacionais para a Educação das Relações Étnico-Raciais e para o Ensino de História e Cultura Afro-Brasileira e Africana. \\ \midrule
    Resolução CNE/MEC nº 3/2002 & Institui	as Diretrizes Curriculares Nacionais Gerais para a organização e o funcionamento dos cursos superiores de tecnologia. \\ \midrule
    Parecer CNE/CES nº 136/2012	& Apresentar as Diretrizes Curriculares Nacionais para os cursos de graduação em Computação. \\ \midrule
    Parecer CNE/MEC nº 3/2004 & Apresentar as Diretrizes Curriculares Nacionais para a Educação das	Relações Étnico-Raciais e para o Ensino de História e Cultura Afro-Brasileira e Africana. \\ \midrule
    Parecer CNE/MEC nº 261/2006	& Dispor sobre procedimentos a serem adotados quanto ao conceito de hora-aula e dá outras providências. \\ \midrule
    Portaria MEC nº 1.428/2018	& Dispõe sobre a oferta, por Instituições de Educação Superior - IES, de disciplinas na modalidade a distância em cursos de graduação presencial. \\
    \bottomrule
    \end{longtable}
\end{scriptsize}      
\end{center}


Vale ressaltar que, em atendimento à Resolução CNE/MEC nº 1/2012\footnote{\url{http://portal.mec.gov.br/dmdocuments/rcp001_12.pdf}}, a Educação em Direitos Humanos será trabalhada de forma transversal no currículo do Curso de Ciência da Computação.

Na busca de promover a educação de cidadãos atuantes e conscientes quanto à pluralidade étnico-racial do Brasil, e, considerando o disposto no Parecer CNE/MEC nº 3/2004\footnote{\url{http://portal.mec.gov.br/dmdocuments/cnecp_003.pdf}}, na Resolução CNE/MEC nº 1/2004\footnote{\url{http://portal.mec.gov.br/cne/arquivos/pdf/res012004.pdf}} e Resolução CEPE/UFRPE nº 217/2012, Art. 2º, será ofertada a disciplina optativa de Educação das Relações Étnico-Raciais para os alunos do curso de graduação em Ciência da Computação.

A inserção dos conhecimentos concernentes à Educação Ambiental ocorrerá de forma integrada e interdisciplinar, obedecendo a Lei nº 9.795/1999\footnote{\url{http://www.planalto.gov.br/ccivil_03/leis/l9795.htm}}, e a Resolução CNE/MEC nº 2/2012\footnote{\url{http://portal.mec.gov.br/dmdocuments/rcp002_12.pdf}}. Além disso, o curso estará atento às diretrizes dos órgãos e sociedades representativas de suas áreas de atuação profissional, como, por exemplo, a \textit{Sociedade Brasileira de Computação} (SBC), que constitui a principal entidade representativa dos profissionais da grande área de computação no Brasil. Destaca-se também que está previsto a oferta da disciplina optativa de Libras, em atendimento ao Decreto no 5.626/2005\footnote{\url{http://www.planalto.gov.br/ccivil_03/_Ato2004-2006/2005/Decreto/D5626.htm}} e a Resolução CEPE/UFRPE no 030/2010.

Vale salientar que as disciplinas da matriz curricular do Curso de Bacharelado em Ciência da Computação poderão ser ofertadas na modalidade semipresencial (EAD). A oferta destas disciplinas não ultrapassará o percentual de 20\% da carga horária total do curso, conforme estabelecido através da portaria do MEC de N° 4.059/04.

Tal como os preceitos outorgados pelos dispositivos legais citados anteriormente, servirão de alicerce para o Curso de Bacharelado em Ciência da Computação as resoluções internas da UFRPE, como se observa no Quadro~\ref*{quadro:base-legal-ufrpe-curso}:

\begin{center}
    
    \begin{scriptsize}
        \begin{longtable}{@{}lp{8.7cm}}
            \caption{\label{quadro:base-legal-ufrpe-curso}Base legal da UFRPE que fundamenta o curso.}\\
    \toprule
    \textbf{Resolução} & \textbf{Escopo} \\
    \midrule
    Resolução CEPE/UFRPE 220/2016 & Revogar a Resolução nº 313/2003 deste Conselho, que regulamentava as diretrizes para elaborar e reformular os Projetos Pedagógicos dos 	Cursos de Graduação da UFRPE e dá outras providências. \\ \midrule
    Resolução CEPE/UFRPE 597/2009 & Revogar a resolução 430/2007 e aprova novo Plano de Ensino, dos 	procedimentos e orientações para elaboração, execução e acompanhamento. \\ \midrule
    Resolução CEPE/UFRPE 217/2012 &	Estabelecer a inclusão do componente curricular ``Educação das Relações Étnico-Raciais'', nos currículos dos cursos de graduação da UFRPE. \\ \midrule
    Resolução CEPE/UFRPE 030/2010 & Estabelecer a inclusão do componente curricular ``LIBRAS'' nos 	currículos dos cursos de graduação da UFRPE. \\ \midrule
    Resolução CEPE/UFRPE 425/2010 &	Regulamentar a previsão nos Projetos Pedagógicos de curso da equiparação das atividades de Extensão, monitorias e iniciação científica como estágios curriculares. \\ \midrule
    Resolução CEPE/UFRPE 065/2011 & Aprovar a criação e regulamentação da implantação do Núcleo Docente Estruturante - NDE dos Cursos de Graduação da UFRPE. \\ \midrule
    Resolução CEPE/UFRPE 003/2017 &	Aprova alteração das Resoluções nº 260/2008 e nº 220/2013, ambas do CONSU da Universidade Federal Rural de Pernambuco. \\ \midrule
    Resolução CEPE/UFRPE 494/2010 & Dispor 	sobre a verificação da aprendizagem no que concerne aos Cursos de Graduação. \\ \midrule
    Resolução CEPE/UFRPE 362/2011 & Estabelece 	critérios para a quantificação e o registro das Atividades Complementares nos cursos de graduação desta Universidade. \\ \midrule
    Resolução CEPE/UFRPE nº 622/2010 & Regulamenta normas de inserção de notas de avaliação de aprendizagem no Sistema de Informações e Gestão Acadêmica – SIG@ da UFRPE. \\ \midrule
    Resolução CEPE/UFRPE nº 678/2010 & Estabelece normas para organização e regulamentação do Estágio Supervisionado Obrigatório para os estudantes dos cursos de graduação da UFRPE e dá outras providências. \\ \midrule
    Resolução CEPE/UFRPE nº 486/2006 & Dispor sobre obrigatoriedade de alunos ingressos na UFRPE de cursarem os dois primeiros semestres letivos dos cursos para os quais se habilitaram. \\ \midrule
    Resolução CEPE/UFRPE nº 154/2001 & Estabelece critérios para desligamento de alunos da UFRPE por insuficiência de rendimentos e discurso de prazo. \\ \midrule
    Resolução CEPE/UFRPE nº 281/2017 & Aprova depósito legal de Monografias e Trabalhos de Conclusão de Cursos de Graduação e Pós-Graduação Lato Sensu da UFRPE.\\
    \bottomrule
    \end{longtable}
\end{scriptsize}     
\end{center}

Além das resoluções descritas no Quadro~\ref{quadro:base-legal-ufrpe-curso}, outras normativas institucionais da UFRPE serão referenciadas ao longo deste projeto.