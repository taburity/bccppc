\chapter{HISTÓRICO DA UFRPE}

A UFRPE é uma instituição centenária, com atuação proeminente no estado de Pernambuco e região. Sua história tem início com a criação das Escolas Superiores de Agricultura e Medicina Veterinária do Mosteiro de São Bento, em Olinda, no dia 3 de novembro de 1912. Apenas em fevereiro de 1914 iniciaram-se as aulas na instituição que, por sua vez, funcionava em um prédio anexo ao Mosteiro, sob a direção do abade alemão D. Pedro Roeser. Em dezembro do mesmo ano foi instalado o Hospital Veterinário, sendo este o primeiro do país (MELO, 2010). Tendo em vista as limitações de espaço para as aulas práticas do curso de Agronomia, os beneditinos transferiram, em 1917, o referido curso para o Engenho São Bento, localizado no distrito de Tapera, em São Lourenço da Mata.

A década de 1930 foi marcada pela estatização da Instituição, com a desapropriação da Escola Superior de Agricultura de São Bento, em 9 de dezembro de 1936, pela Lei nº 2.443 do Congresso Estadual e Ato nº 1.802 do Poder Executivo Estadual, passando a denominar-se Escola Superior de Agricultura de Pernambuco (ESAP). Pouco mais de um ano depois, através do Decreto nº 82, de 12 de março de 1938, ela foi transferida para o Bairro de Dois Irmãos, no Recife.

Em 1947, através do Decreto Estadual nº 1.741, foram reunidos a ESAP, o Instituto de Pesquisas Agronômicas, o Instituto de Pesquisas Zootécnicas e o Instituto de Pesquisas Veterinárias, constituindo, assim, a Universidade Rural de Pernambuco (URP). Em 1955, através da Lei Federal nº 2.524, a Universidade foi federalizada, passando a fazer parte do Sistema Federal de Ensino Agrícola Superior vinculado ao Ministério da Agricultura. Após a federalização, a URP elaborou o seu primeiro estatuto, em 1964, com base na LDB de 1961. Com a promulgação do Decreto Federal nº 60.731, de 19 de maio de 1967, a instituição passou a denominar-se oficialmente \textit{Universidade Federal Rural de Pernambuco} (UFRPE).

Em 1957, a Escola Agrotécnica do Nordeste foi incorporada à Universidade passando a ser denominada, a partir de 1968, de Colégio Agrícola Dom Agostinho Ikas (SOUZA, 2000). Atualmente, o Colégio, que também conta com um novo campus em Tiúma, oferece cursos técnicos em Agropecuária (integrado ou não ao Ensino Médio), Alimentos e Administração, além de ofertar outros na modalidade de Educação a Distância (EAD): Açúcar e Álcool, Alimentos e Administração. Também é destaque sua atuação no âmbito da qualificação profissional, por meio do Programa Nacional de Acesso ao Ensino Técnico e Emprego, tendo formado, desde 2013, mais de 12.000 estudantes em todas as regiões do estado de Pernambuco.

Na década de 1970, novos cursos de graduação foram criados, sendo eles: Estudos Sociais, Zootecnia, Engenharia de Pesca, Bacharelado e Licenciatura em Ciências Biológicas, Economia Doméstica, Ciências Agrícolas, Engenharia Florestal, Matemática e Química. No mesmo período, a UFRPE iniciou suas atividades de oferta de curso de pós-graduação \textit{stricto sensu}, com a criação do Mestrado em Botânica, em 1973, por meio de um convênio firmado com a Universidade Federal de Pernambuco (UFPE).

Os anos de 1980 se destacaram pela reformulação do curso de Licenciatura em Ciências com suas respectivas habilitações. Surgiram, então, quatro novos cursos de Licenciatura Plena: Física, Química, Matemática e Ciências Biológicas.

Nos anos 2000, a UFRPE vivenciou a expansão de suas atividades com a criação de cursos de graduação (na Sede) e das Unidades Acadêmicas, através do Programa de Reestruturação e Expansão das Universidades Federais. A Unidade Acadêmica de Garanhuns (UAG), localizada no Agreste de Pernambuco, foi a primeira das unidades fundadas pela UFRPE, tendo iniciado suas atividades no segundo semestre de 2005. A UAG oferta os cursos de Agronomia, Licenciatura em Pedagogia, Letras, Ciência da Computação, Engenharia de Alimentos, Medicina Veterinária e Zootecnia. Destaque-se que a UAG está em processo de emancipação, devendo, em alguns anos, tornar-se uma instituição autônoma. Em 2006, no Sertão de Pernambuco, foi criada a Unidade Acadêmica de Serra Talhada (UAST) que, atualmente, oferta os cursos de Bacharelado em: Administração, Ciências Biológicas, Ciências Econômicas, Sistemas de Informação, além de Engenharia de Pesca, Agronomia, Licenciatura em Letras, Licenciatura em Química e Zootecnia.

Ainda no processo de expansão e inclusão social, em 2005, através do Programa Pró-Licenciatura do Ministério da Educação, a UFRPE iniciou as atividades do ensino de graduação na modalidade à distância. Em 2006, o MEC implantou o Programa Universidade Aberta do Brasil, cuja prioridade foi a formação de profissionais para a Educação Básica. Nesse mesmo ano, a Universidade se engajou no referido programa. Em 2010, foi criada a Unidade Acadêmica de Educação a Distância e Tecnologia (UAEADTec), presente em 19 polos nos estados de Pernambuco e Bahia. Sua sede administrativa está localizada no \textit{campus} Dois Irmãos, no Recife. A UAEADTec oferta oito cursos de graduação: Bacharelado em Administração Pública, Bacharelado em Sistemas de Informação, Licenciatura em Artes Visuais Digitais, Licenciatura em Computação, Licenciatura em Física, Licenciatura em História, Licenciatura em Letras e Licenciatura em Pedagogia.

Ao mesmo tempo em que essa interiorização vem se consolidando com a oferta de cursos presenciais e a distância, a UFRPE também inovou, em 2014, com a implementação da Unidade Acadêmica no Cabo de Santo Agostinho (UACSA). A referida Unidade tem ofertado tanto cursos Superiores em Tecnologia (Construção Civil, Transmissão e Distribuição Elétrica, Automação Industrial, Gestão da Produção Industrial, Mecânica: Processos Industriais) quanto de Bacharelado em Engenharia (Civil, Elétrica, Eletrônica, Materiais e Mecânica).

Em 2017, o Conselho Universitário da UFRPE, através da Resolução CONSU/ UFRPE nº 098/2017\footnote{\url{http://seg.ufrpe.br/resolucao/res-n\%C2\%BA-0982017}}, aprovou a criação da Unidade Acadêmica de Belo Jardim (UABJ), visando atender às demandas de qualificação profissional nas áreas de Engenharia da região. De forma semelhante ao projeto da UACSA, a UABJ ofertará cursos Superiores em Tecnologia (Eletrônica Industrial, Redes de Computadores, Processos Químicos, Gestão de Recursos Hídricos) e de Bacharelado em Engenharia (Controle e Automação, Computação, Química e Hídrica).

A Universidade Federal do Agreste de Pernambuco (UFAPE) tem sua origem no ano de 2018, a partir da Lei nº 13.651, de 11 de abril de 2018, através do desmembramento da Universidade Federal Rural de Pernambuco (UFRPE) / Unidade Acadêmica de Garanhuns (UAG). Em seguida, teve início a vigência do Termo de Colaboração Técnica, celebrado entre o Ministério da Educação (MEC), por intermédio da Secretaria de Educação Superior, e a UFRPE, para a implantação da UFAPE.


