\chapter{CAMPO DE ATUAÇÃO PROFISSIONAL}

Na contemporaneidade tem-se exigido respostas céleres a problemas complexos decorrentes do mundo globalizado, no qual a informação adquire um papel proeminente. Não é por acaso que o atual modo de vida das pessoas está intrinsecamente ligado ao uso das tecnologias, em especial, dos computadores. Estes podem ser encontrados nos mais variados lugares, como, por exemplo, nos lares (em TV's, eletrodomésticos, vídeo games), escolas (PC's, tablets, laboratórios), indústria (equipamentos de segurança, relógios-ponto, máquinas), comércio (caixas registradoras), dentre outros. 

Desta forma, o profissional atuará possivelmente nos seguintes problemas:

\begin{itemize}
    \item concepção, especificação, projeto, construção, avaliação e adaptação de sistemas digitais;
    \item análise e projeto de estrutura lógica e funcional de computadores e sua implementação;
    \item desenvolvimento e implementação de software básico e de apoio para sistemas computacionais;
    \item projeto e desenvolvimento de sistemas e programas usando linguagens de programação;
    \item projeto e desenvolvimento de sistemas de estruturação de informação;
    \item projeto e desenvolvimento de redes de processamento local e remota, em matéria de hardware e de software.    
\end{itemize}

O egresso do curso de Bacharelado em Ciência da Computação deve estar preparado para propor soluções inovadoras e adequadas para problemas propostos, capacitado a acompanhar e avaliar avanços tecnológicos em computação, bem como aplicar e implementar as evoluções, reposições e adaptações que se façam necessárias, tanto de forma reativa como pró-ativa, logo deve estar apto a desenvolver as seguintes funções no mercado de trabalho:

\begin{itemize}
    \item \textbf{Empreendedor} – descobrimento e empreendimento de novas oportunidades para aplicações usando sistemas computacionais e avaliando a conveniência de se investir no desenvolvimento da aplicação;
    \item \textbf{Consultor} – consultoria e assessoria a empresas de diversas áreas no que tange ao uso adequado de sistemas computacionais;
    \item \textbf{Coordenador de equipe} – coordenação de equipes envolvidas em projetos na área de computação e informática;
    \item \textbf{Membro de equipe} – participação de forma colaborativa e integrada de equipes que desenvolvem projetos na área de informática;
    \item \textbf{Pesquisador} – participação em projetos de pesquisa científica e tecnológica. 
\end{itemize}