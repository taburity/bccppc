\chapter{REQUISITOS DE INGRESSO}

O curso de Bacharelado em Ciência da Computação tem duas entradas anuais com 40 vagas por semestre letivo, resultando em 80 vagas por ano. O ingresso dos alunos ocorre através do Sistema de Seleção Unificado – SISU, com base nos resultados obtidos no Exame Nacional do Ensino Médio – ENEM, e do Ingresso Extra.

\begin{enumerate}
    \item \textbf{Ingresso através do ENEM}: a UFRPE adota o SISU como principal meio de acesso aos cursos de graduação, através da nota do ENEM, considerando as duas entradas semestrais.
    \item \textbf{Ingresso Extra}: além do ingresso semestral, a partir da seleção do SISU, a UFRPE possui outras modalidades de acesso. Estas ocorrem duas vezes por ano, em datas previstas e com editais publicados pela Pró-Reitoria de Ensino de Graduação – PREG. 
\end{enumerate}

Nessa direção, são modalidades de ingresso extra:

\begin{itemize}
    \item \textbf{Reintegração} – Após ter perdido o vínculo com a Universidade, o aluno que tenha se evadido pelo período máximo de integralização de seu curso poderá requerer a reintegração, uma única vez, no mesmo curso (inclusive para colação de grau), desde que tenha condições de concluí-lo no prazo máximo permitido (considerando o prazo do vínculo anterior e o que necessitará para a integralização do currículo) e que não possua 4 (quatro) ou mais reprovações em uma mesma disciplina (Fundamentação: Res. CEPE/UFRPE nº 100/83 (de 16 de setembro de 1983) e Resolução CEPE/UFRPE nº 354/2008 (de 13 de junho de 2008).
    \item \textbf{Reopção ou Transferência Interna} – O aluno regularmente matriculado que esteja insatisfeito com o seu curso poderá requerer a transferência interna para outro curso de graduação desta Universidade. Para tanto, ele deverá considerar: a área de conhecimento afim ao seu curso de origem; a existência de vagas no curso pretendido; o cumprimento de, no mínimo, 40\% (quarenta por cento) do currículo original do seu curso, dispondo, portanto, de tempo para integralização curricular, considerando os vínculos com o curso anterior e o pretendido (Fundamentação: Res. CEPE/UFRPE nº 34/97, de 16/01/1997).
    \item \textbf{Transferência Externa} – A Universidade recebe alunos de outras IES, vinculados a cursos reconhecidos pelo CNE, desde que eles: desejem continuar o curso iniciado ou ingressar em curso de área afim; estejam com vínculo ativo ou trancado com a Instituição de origem; tenham condições de integralizar o currículo no seu prazo máximo, considerando, também, o prazo definido pela outra IES e o que necessitaria cursar na UFRPE; e, por fim, que tenham cursado todas as disciplinas constantes do primeiro período da matriz curricular do curso pretendido na UFRPE. Salvo os casos de transferência ex-officio (que independem de vagas), é necessário, para ingresso, que o curso tenha vagas ociosas (Fundamentação: Res. CEPE/ UFRPE nºs 124/83 e 180/91).
    \item \textbf{Portadores de Diploma de Curso Superior} – Os portadores de diploma de curso superior, reconhecido pelo CNE, que desejem realizar matrícula em outro curso superior na UFRPE, em área afim, podem requerê-la, desde que haja disponibilidade após o preenchimento de vagas pelas demais modalidades de ingresso. (Fundamentação: Res. CEPE/UFRPE nº 181/91, de 01/10/1991).
\end{itemize}

As formas de ingresso definidas a seguir independem de vagas e não há necessidade de publicação de edital da PREG:

\begin{itemize}
    \item \textbf{Cortesia Diplomática} – Em atendimento ao que preconiza o Decreto nº 89.758/84, de 06/06/84, a UFRPE aceita alunos incluídos nas seguintes situações: funcionário estrangeiro, de missão diplomática ou repartição consular de carreira no Brasil, e seus dependentes legais; funcionário estrangeiro de Organismo Internacional que goze de privilégios e imunidades em virtude de acordo entre o Brasil e a organização, e seus dependentes legais; técnico estrangeiro, e seus dependentes legais, que preste serviço em território nacional, no âmbito de acordo de cooperação cultural, técnica, científica ou tecnológica, firmado entre o Brasil e seu país de origem, desde que em seu contrato esteja prevista a permanência mínima de 1 (um) ano no Brasil; e, finalmente, técnico estrangeiro, e seus dependentes legais, de organismo internacional, que goze de privilégios e imunidades em virtude de acordo entre o Brasil e a organização, desde que em seu contrato esteja prevista a permanência mínima de 1 (um) ano em território nacional.

    Este tipo de ingresso nos cursos de graduação se dá mediante solicitação do Ministério das Relações Exteriores, encaminhada pelo MEC, com a isenção de processo seletivo e independentemente da existência de vagas, sendo, todavia, somente concedido a estudantes de países que assegurem o regime de reciprocidade e que sejam portadores de visto diplomático ou oficial.

    \item \textbf{Programa de Estudantes-Convênio de Graduação (PEC-G)} – Alunos provenientes de países em desenvolvimento, especialmente da África e da América Latina, são aceitos como estudantes dos cursos de graduação da UFRPE. Estes estudantes são selecionados, por via diplomática em seus países, considerando os mecanismos previstos no protocolo do PEC-G e obedecendo aos princípios norteadores da filosofia desse Programa. Não pode ser admitido, através desta modalidade, o estrangeiro portador de visto de turista, diplomático ou permanente, bem como o brasileiro dependente dos pais que, por qualquer motivo, estejam prestando serviços no exterior, e o indivíduo com dupla nacionalidade, sendo uma delas brasileira.
    \item {\bfseries Transferência Obrigatória ou \textit {ex officio}} – É a Transferência definida na Lei n.º 9.536, de 11/12/97 que regulamenta o Art. 49 da Lei n.º 9.394, de 20/12/96, Portaria Ministerial nº 975/92, de 25/06/92 e Resolução nº 12, de 02/07/94 do Conselho Federal de Educação - CFE. Esta transferência independe da existência de vaga e época, abrangendo o servidor público federal da administração direta ou indireta, autárquica, fundacional ou membro das Forças Armadas, regidos pela Lei n.º 8.112/90, inclusive seus dependentes, quando requerido em razão de comprovada remoção ou transferência \textit {ex officio}. A transferência deverá implicar em mudança de residência para o município onde se situe a instituição recebedora ou para localidade próxima a esta, observadas as normas estabelecidas pelo CNE.
\end{itemize}